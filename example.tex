\documentclass{beamer}
\usepackage[utf8x]{inputenc}
\usepackage{ucs}
\usepackage[english]{babel}
\usepackage{amsmath}
\usepackage{epstopdf}
\usepackage{lmodern}
\usetheme{kth}
\setlength\parindent{0pt}
\date{DD-MM-YYYY}
\title{Title}
\subtitle{Subtitle}
\author{Name Surname}

\begin{document}
\begin{frame}[plain]
% Would be nice to not have to use [plain] here
\maketitle
\end{frame}


\addFootnote{Grimler H. Some publication}
\addFootnote{Grimler H. Another publication}
% Would be nice to not have to specify footnotes before \begin{frame}
\begin{frame}{Features of the theme}
  The theme uses native beamercolorboxes under the hood.
  \begin{itemize}
    \item This means that background/font-colors can be changed using \textbackslash setbeamercolor\{\}\{\}
    \item The colored theme parts are made with tikz instead of included as
        images. Color can therefore relatively easy be changed (TODO: make
        theme accept color as option)
    \item The drawn parts are ``protected'' from \textbackslash
      tikzexternalize to prevent lots of generated banner images if using the
      tikzexternal library (see
      \href{https://tex.stackexchange.com/questions/295885}
           {tex.stackexchange.com/questions/295885})
    \item References can be added in the footer using the command
      \textbackslash addFootnote\{Grimler H. Some publication\} $^{1,2}$
\end{itemize}
\vfill
\end{frame}

{
  \setbeamercolor{footer}{bg = kthred}
  \setbeamercolor{itemize item}{fg = kthred}
  \begin{frame}{Other colors}
    \begin{itemize}
      \item The colors on this slide have been changed with \textbackslash
        setbeamercolor\{footer\} and \textbackslash setbeamercolor\{itemize item\}\{fg = kthred\}
    \end{itemize}
  \end{frame}
}
\end{document}
